\chapter{Configuration}\index{Configuration}

This chapter will explain how to configure WendzelNNTPd once it has been installed on your system.

{\bf Note:} The configuration file of WendzelNNTPd is named {\it /usr/local/etc/wendzelnntpd.conf}. The format of the configuration file should be self describing and the standard config file includes many comments which will help you to understand the lines you can see and modify there.

{\bf Note:} On *nix-like operating systems the default installation path is {\it /usr/local/*} what means that the configuration file of WendzelNNTPd will be {\it /usr/local/etc/wendzelnntpd.conf}, and the binaries will be placed in {\it /usr/local/sbin}. %On Windows all files are placed in the directory you choosed for the installation.

\section{Choosing an database engine}

The first, and most important step, is to choose a database engine. You can either use SQlite3 (what is the default case and easy to use, but not very performant) or MySQL (what is of course the better solution, but also a little bit more complicated to realize). By default, WendzelNNTPd is configured for SQlite3 and is ready to run. If you want to keep this setting, you do not have to read this section.

\subsection{Modifying the wendzelnntpd.conf}

In your configuration file, you will find a line called {\bf database-engine}. Here you can either write {\bf sqlite} or {\bf mysql}.

\begin{verbatim}
database-engine mysql
\end{verbatim}

If you use MySQL, you also need to specify the user and password which WendzelNNTPd should use to connect to the server. If your server does not run on the localhost or uses not the default MySQL port, you have to modify these values too.

\begin{verbatim}
; Your database hostname (not needed for sqlite3)
database-server 127.0.0.1

; the database connection port (not needed for sqlite3)
; Comment out to use the default port of your database engine
database-port 3306

; Server authentication (not needed for sqlite3)
database-username mysqluser
database-password supercoolpass
\end{verbatim}

\subsection{Generating your database tables}

Once you have chosen your database, you need to create the (database and) table in your database.

\subsubsection{SQlite}

In the case of SQlite, {\bf make install} already did this for you.

{\bf Note:} The SQlite database file as well as the posting management files will be stored in {\it /var/spool/news/wendzelnntpd/}.

\subsubsection{MySQL}

For MySQL, a SQL script file is included called {\it mysql\_db\_struct.sql}. It creates the WendzelNNTPd database as well as all needed tables. Use the MySQL console tool to execute the script.

\begin{verbatim}
$ cd /path/to/your/extracted/wendzelnntpd-archive/
$ mysql -u YOUR-USER -p
Enter password: 
Welcome to the MySQL monitor.  Commands end with ; or \g.
Your MySQL connection id is 48
Server version: 5.1.37-1ubuntu5.1 (Ubuntu)

Type 'help;' or '\h' for help. Type '\c' to clear the current input statement.

mysql> source mysql\_db\_struct.sql
...
mysql> quit
Bye
\end{verbatim}

\section{Network Settings}

Now you should specify the IP addresses (IPv4 or IPv6) for WendzelNNTPd to accept connections on, as well as the TCP port (the NNTP default port is 119) to run on.

\begin{verbatim}
; You have to specify the port _before_ using the 'listen' command!
port	119
; network addresses to listen on
listen	127.0.0.1
listen	::1
listen	192.168.0.1
\end{verbatim}

You {\it could} also use different ports for different IP addresses by placing a {\bf port} command right before each {\bf listen} command but this is not recommended.

\section{Verbose Mode}

If you have any problems running WendzelNNTPd or if you simply want more information about what is happening, you can uncomment the {\bf verbose-mode} line.

\begin{verbatim}
; Uncomment 'verbose-mode' if you want to find errors or if you
; have problems with the logging subsystem. All log strings are
; written to stderr too, if verbose-mode is set. Additionaly all
; commands sent by clients are written to stderr too (but not to
; logfile)
verbose-mode
\end{verbatim}

\section{Security Settings}

\subsection{Authentication and Access Control Lists (ACL)}

WendzelNNTPd contains an extensive access control subsystem. If you want to allow only authenticated users the access to the server, you should uncomment {\bf use-authentication}. This gives every authenticated user access to each newsgroup.

\begin{verbatim}
; Activate authentication
use-authentication
\end{verbatim}

If you need a really advanced authentication system, you can activate Access Control Lists (ACL) by uncommenting {\bf use-acl}. This activates the support for Role based ACL too.

\begin{verbatim}
; If you activated authentication, you can also activate access
; control lists (ACL)
use-acl
\end{verbatim}

\subsection{Anonymized Message-ID}

By default, WendzelNNTPd makes a users hostname or IP part of new message-IDs when a user sents a posting using the NNTP POST command. If you do not want that, you can force WendzelNNTPd not to do that by uncommenting {\bf enable-anonym-mids}, what enables anonymized Message-IDs.

\begin{verbatim}
; This prevents that IPs or Hostnames will become part of the
; message ID generated by WendzelNNTPd what is the default case.
; Uncomment it to enable this feature.
enable-anonym-mids
\end{verbatim}


