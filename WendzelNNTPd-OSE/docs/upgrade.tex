\chapter{Upgrade to a newer version}\index{Upgrade}\label{Ch:Upgrade}

\section{Upgrade from 2.0.x to 2.0.y}

Stop WendzelNNTPd if it is currently running. Install WendzelNNTPd as described but run {\bf make upgrade} instead of {\bf make install}. Afterwards, start WendzelNNTPd again.

~

~

\section{Upgrade from v.1.4.x to 2.0.x}

{\bf Acknowledgement:} I would like to thank Ann from Href.com for helping a lot with finding out how to upgrade from 1.4.x to 2.0.x!

~

An upgrade from version 1.4.x was not foreseen due to the limited available time I have for the development. However, here is a dirty hack:

\paragraph*{First Step:} You need to install Wendzelnntpd-2.x on a Linux system. This requires some libraries and tools.
Under {\it Ubuntu} they all come as packages: {\bf sudo apt-get install libmysqlclient-dev, libsqlite3-dev, flex, bison, sqlite3}. Under {\it CentOS} they come as packages as well: {\bf sudo yum install make gcc bison flex sqlite-devel}. {\it Other} operating systems should provide the same or similar packages/ports.

Run {\bf MYSQL=NO ./configure}, {\bf make}, and {\bf sudo make install}. This will compile, build and install WendzelNNTPd without MySQL support as you only rely on SQLite3 from v.1.4.x.

\paragraph*{Second Step:} Please make sure WendzelNNTPd-2 is configured in a way that we can *hopefully* make it work with your own database file. In {\bf /usr/local/etc/wendzelnntpd.conf} make sure that WendzelNNTPd uses sqlite3 instead of mysql:

\begin{verbatim}
database-engine sqlite3
\end{verbatim}

\paragraph*{Third Step:}
Now comes the tricky part. The install command should have created
{\bf /var/spool/news/wendzelnntpd/usenet.db}
However, it is an empty usenet database file in the new format.
Now REPLACE that file with the file you use on your existing WendzelNNTPd installation, which uses the old 1.4.x format. Also copy all {\bf cdp*} files and the {\bf nextmsgid} file from your Windows directory to {\bf /var/spool/news/wendzelnntpd/}. 

The following step is a very dirty hack but I hope it works for you. It is not 100\% perfect as important table columns are then still of the type 'STRING' instead of the type 'TEXT'!

Load the sqlite3 tool with your replaced database file:

\begin{verbatim}
$ sudo sqlite3 /var/spool/news/wendzelnntpd/usenet.db
\end{verbatim}

This will open the new file in editing mode. We now add the tables which are not part of v.1.4.x to your existing database file. Therefore enter the following commands:

\begin{verbatim}
CREATE TABLE roles (role TEXT PRIMARY KEY);
CREATE TABLE users2roles (username TEXT, role TEXT, PRIMARY KEY(username, role));
CREATE TABLE acl_users (username TEXT, ng TEXT, PRIMARY KEY(username, ng));
CREATE TABLE acl_roles (role TEXT, ng TEXT, PRIMARY KEY(role, ng));
.quit
\end{verbatim}


\paragraph*{Fix Postings}

You will probably see no posting bodies right now if postings are requested by your client. Therefore, switch into {\bf /var/spool/news/wendzelnntpd} and run (as superuser) this command, it will replace the broken trailings with corrected ones:

\begin{verbatim}
$ cd /var/spool/news/wendzelnntpd 
$ for filn in `/bin/ls cdp*`; do echo $filn; cat $filn | \
sed 's/\.\r/.\r\n/' > new;  num=`wc -l new| \
awk '{$minone=$1-1; print $minone}'` ; \
head -n $num new > $filn; done
$
\end{verbatim}

\paragraph*{Last Step (Checking whether it works!):}

First check, whether the datbase file is accepted at all:

\begin{verbatim}
$ sudo wendzelnntpadm listgroups
\end{verbatim}

It should list all your newsgroups

\begin{verbatim}
$ sudo wendzelnntpadm listusers
\end{verbatim}

It should list all existing users. Accordingly

\begin{verbatim}
$ sudo wendzelnntpadm listacl
\end{verbatim}

should list all access control entries (which will be empty right now but if no error message appears, the related tables are now part of your database file!).

~

Now start WendzelNNTPd via {\bf sudo wendzelnntpd} and try to connect with an NNTP client to your WendzelNNTPd and then try reading postings and try sending new postings and try replying to these.

If this works, you can now run v2.x on 32bit and 64bit Linux :)

